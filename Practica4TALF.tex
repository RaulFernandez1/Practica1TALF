\documentclass{article}
% pre\'ambulo

\usepackage[utf8]{inputenc}
\usepackage[spanish]{babel}
\usepackage{amsthm, amsmath}
\usepackage{nccmath}
\usepackage{graphicx}
\usepackage{enumitem}
\usepackage{caption}
\usepackage{subcaption}
\usepackage{whilecode2}

\graphicspath{ {/home/alumno/Descargas/Practica4} }
\newtheorem{example}{Paso}
\newtheorem{definicion}{Definición}[section]

\title{Teoría de Autómatas y Lenguajes Formales\\[.4\baselineskip]Práctica 4: EXWHILE y Codificación de Programas}
\author{Raul, Fernandez Escaño}
\date{\today}

\begin{document}
% cuerpo del documento

\maketitle

\section{Programas WHILE}
\begin{flushleft}Para obtener el programa, he ido probando y buscando cual es la configuración adecuada para que este diverja. A continuación, se les mostrará cual es:\end{flushleft}

\whileprogram{Q}{0} {
	\DefaultVar{2}\Assig\DefaultVar{1} + 1; \\
	\While{$X_2 \not = 0$} {
		\DefaultVar{1}\Assig 0 \\
	} 
}{s}

\section{Vectores usando Octave}

\begin{flushleft}En este caso, se creará un script en Octave que permita decodificar dada una entrada N todos los vectores cuya codificación sea N o inferior. El script será el siguiente: \end{flushleft}
function allVectorsN(N) \\
  for i=0:N-1 \\
    disp(['(' num2str(godeldecoding(i)) ')']) \\
  end \\
end

\begin{flushleft}Dado este código, a continuación se han probado las siguientes entradas:\end{flushleft}

\includegraphics[scale=0.30]{todosVectores}

\section{WHILE usando Octave}

\begin{flushleft}En este caso, se creará un script en Octave que permita decodificar dada una entrada N todos los programas WHILE cuya codificación sea N o inferior. El script será el siguiente: \end{flushleft}
function allWhileProgramsN(N) \\
  for i=0:N-1 \\
    disp(N2WHILE(i)) \\
  end \\
end

\begin{flushleft}Dado este código, a continuación se han probado las siguientes entradas:\end{flushleft}

\includegraphics[scale=0.30]{todosProgramasWhile}

\end{document}