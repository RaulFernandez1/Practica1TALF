\documentclass{article}
% pre\'ambulo

\usepackage[utf8]{inputenc}
\usepackage[spanish]{babel}
\usepackage{amsthm, amsmath}
\usepackage{nccmath}
\usepackage{graphicx}
\usepackage{enumitem}
\newtheorem{example}{Paso}

\title{Teoría de Autómatas y Lenguajes Formales\\[.4\baselineskip]Práctica 1: Latex y expresiones regurales}
\author{Raul, Fernandez Escaño}
\date{11 de octubre de 2022}

\begin{document}
% cuerpo del documento

\maketitle

\section{Potencia de un conjunto}
Encuentra la potencia cúbica del conjunto $R=\{(1,1),(1,2),(2,3)
,(3,4)\}$. En todo momento, se tendrá en cuenta esta fórmula:
\begin{equation*}
 R^n = 
 \begin{cases}
 	R & n = 1 \\
 	\{(a,b)\colon \exists c \in A, (a,c)\in R^{n-1} \land
 	(c,b) \in R \} & n > 1
 \end{cases}
\end{equation*}
 \begin{example}
 Dado que buscamos la potencia cúbica, antes debemos calcular la potencia cuadrada. Sustituyendo nuestro valor por la n nos sale que:
  \begin{equation*}
 	R^2 = \{(a,b)\colon \exists c \in A, (a,c)\in R \land 
 	(c,b)\in R \} = \{(1,1),(1,2),(1,3),(2,4)\}
  \end{equation*}
 Como podemos observar, $(1,1)\in R$ se puede relacionar consigo mismo ya que $\exists c \in A$, siendo este el 1 y dando a lugar al par $(1,1)$ que pertenece a $R^2$. Así se haría con el resto de elementos de R.
 \end{example}
 \begin{example}
 Al calcular $R^2$, podemos realizar el mismo cálculo para obtener $R^3$, que será en este caso la solución para nuestro problema.
  \begin{equation*}
 	R^3 = \{(a,b)\colon \exists c\in A, (a,c)\in R^2 \land
 	(c,b) \in R\} = \{(1,1),(1,2),(1,3),(1,4)\}
  \end{equation*}
  En este caso, las relaciones se establecen entre $R^2$ y $R$.
  Un ejemplo para la obtención de algun par es el caso de 
  $(1,4)$, que se obtiene a partir de $(1,3)\in R^2$ y de
  $(3,4)\in R$ siendo $c = 3$
 \end{example}
\end{document}