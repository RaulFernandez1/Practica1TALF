\documentclass{article}
% pre\'ambulo

\usepackage[utf8]{inputenc}
\usepackage[spanish]{babel}
\usepackage{amsthm, amsmath}
\usepackage{nccmath}
\usepackage{graphicx}
\usepackage{enumitem}
\usepackage{caption}
\usepackage{subcaption}
\graphicspath{ {/home/alumno/Descargas/Practica2} }
\newtheorem{example}{Paso}
\newtheorem{definicion}{Definición}[section]

\title{Teoría de Autómatas y Lenguajes Formales\\[.4\baselineskip]Práctica 2: Automatas Finitos}
\author{Raul, Fernandez Escaño}
\date{\today}

\begin{document}
% cuerpo del documento

\maketitle

\section{Automata DFA}
Se considera un lenguaje L sobre un alfabeto $\Sigma=\{a,b\}$ que contiene a la cadena a. Construye un automata DFA que reconozca el lenguaje L, rechazando todas aquellas que no sean válidas
\begin{definicion}(\textbf{Automata DFA}): Un automata finito determinista es una 5-tupla de (K,$\Sigma$,$\delta$,s,F) que cumple:
\end{definicion}
	\begin{enumerate}[label=]
		\item K es un conjunto no vacío
		\item $\Sigma$ es un alfabeto
		\item $s\in K$ es el estado inicial
		\item $F \subseteq K$ es un conjunto de estados finales
		\item $\delta \colon K \times \Sigma \rightarrow K$ es la funcion
		de transición
	\end{enumerate}
 \begin{example}(\textbf{Caracteristicas del automata}) \end{example}
 El automata que se ha definido posee las siguientes caracteristicas:
  \begin{enumerate}
  	\item K = $\{q0,q1,q2\}$
  	\item $\Sigma = \{a,b\}$
  	\item s = q0
  	\item F =$ \{q1\}$
  	\item $\delta = \{(q0,a,q1),(q0,b,q2),(q1,a,q2),(q1,b,q2)
  	,(q2,a,q2),(q2,b,q2)\}$
  \end{enumerate}
  \begin{example}(\textbf{Representacion del automata}) \end{example}Una vez definido las caracteristica de nuestro automata, se recreara    utilizando la herramienta JFLAP, que gracias a su interfaz, se podra 	   apreciar el automata mucho mejor, dando por consiguiente la representacion adjuntada [Figura 1]
\begin{example}(\textbf{Automata en Oracle}) \end{example}Tambien se podrá representar utilizando codificación de Oracle [Figura 2]


\begin{figure}
 	\includegraphics[scale=0.4]{DFA_Image}
 	\caption{}
\end{figure}
\begin{figure}
	\includegraphics{DFA_Code}
	\caption{}
\end{figure}

\end{document}